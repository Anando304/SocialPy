\documentclass[12pt, titlepage]{article}

\usepackage{booktabs}
\usepackage{tabularx}
\usepackage{hyperref}
\usepackage{float}
\usepackage{geometry}
\usepackage{graphicx}
\geometry{a4paper, portrait, margin=1in}
\hypersetup{
    colorlinks=true,
    linkcolor=red,
    urlcolor=cyan,
}

% Different Coloured Text and Strikethrough
\usepackage{xcolor}
\usepackage{ulem}



\title{SE 3XA3: Development Plan\\ Terminal Social Media App}

\author{L03 Team 1
		\\ Anando Zaman, zamana11
        \\ Graeme Woods, woodsg1
        \\ Yuvraj Randhawa, randhawy
        \\ Due February 5, 2021
        \\ Tag DP-Rev.1
}

\date{}


\begin{document}

\begin{table}[hp]
\caption{Revision History} \label{TblRevisionHistory}
\begin{tabularx}{\textwidth}{llX}
    \toprule
    \textbf{Date} & \textbf{Developer(s)} & \textbf{Change}\\
    \midrule
        February 1, 2021 & Graeme & Team Roles, POC, Technology, Style Guide \\
        February 1, 2021 & Anando & Team Roles, Git, POC, Technology \\
        February 1, 2021 & Yuvraj & Team Roles, Git, Technology \\
        \bottomrule
    \end{tabularx}
\end{table}

\newpage

\maketitle

\noindent This document outlines the Development Plan Lab section 3, team 1 project - Terminal Social Media App. This document outlines the team meeting plan, team communication plan, team member roles, Git workflow plan, proof of concept demonstration plan, technology, coding style, project schedule, and project review.

\section{Team Meeting Plan}
% When, Where, Frequency, Roles, Rules for Agenda


\subsection{{Meeting Agenda \& Roles}}
Our team will meet virtually \sout{once or twice a week} \textcolor{red}{on monday and wednesday each week} via FB messenger or MS Teams. Each meeting will be approximately one hour and will take place at 5pm. Each meeting will have its own agenda and each task will be completed by the user assigned that task. Most of the meeting plan and decisions will be recorded via Teams built-in recorder or FB. These meetings will discuss several major components such as blockers to current timelines, technical or non-technical discussions amongst team members, and potential technical demonstrations between different team members process so that everyone is aware. \textcolor{red}{Roles will be rotational with Anando taking charge of the first meeting followed by Yuvraj for the next and Graeme for the last. This cycle will repeat.}

\begin{table}[H]
    \centering
    \caption{{Meeting Roles}}
    \vspace{5pt}
    \begin{tabular}{|p{0.2\textwidth}|p{0.7\textwidth}|}
        \hline
        \textbf{{Roles}} & \textbf{{Description}}\\
        \hline
        Meeting Lead & {Responsible for keeping the meeting on track and ensuring that the meeting is organized according to a meeting agenda} \\
        \hline
        Recorder & {Takes notes of decisions and action items that have been reached during the meeting}\\
        \hline
        Summarizer & {Compiles the meeting notes into meeting minutes} \\
        \hline
        Participant & {Responsible for contributing towards the meeting discussion} \\
        \hline
    \end{tabular}
    
    \label{meetingRoles}
\end{table}

\section{Team Communication Plan}
As the course is completely virtual, the communication outside of class hours will be held mainly through Facebook Messenger. Through Messenger, team members can communicate about different tasks as well as brainstorm any ideas. This way, each team member is imformed and avoids knowledge gaps. As an alternate option, team members can discuss through MS Teams if they are unable to reach via Messenger for technical issues. A Gantt Chart will be used to organize timelines for different aspects of the work with information regarding what aspects have not begun, have completed, and are in-progress.

\newpage
\section{Team Member Roles}
The following table outlines the roles that each team member will be responsible for the course of this project.

\begin{table}[H]
    \centering
    \caption{Team Member Roles}
    \vspace{5pt}
    \begin{tabular}{|l|l|}
        \hline
        \textbf{Role} & \textbf{Member(s)} \\
        \hline
        Project Lead & Anando\\
        \hline
        Developers & Anando, Graeme, Yuvraj \\
        \hline
        Documentation Experts &  Anando, Graeme, Yuvraj\\
        \hline
        Git Expert & Graeme \\
        \hline
        LaTeX Expert & Yuvraj \\
        \hline
        Technology Expert &  Anando \\
        \hline
    \end{tabular}
\end{table}

\section{Git workflow plan}
The Git Workflow will consist of a master branch initially where all team members will stage commits and make changes. This will be a centralized approach and will likely be effective for the initial documentation and design phases since the team is small, and is unlikely to conflict. In the development stages of the project, team members can move to a branched approach if merge conflicts arise or if commits become difficult to track or manage.


\section{Proof of concept demonstration plan}

The main challenges for this project will consist of the Firebase API libraries, porting the application from Java to Python, and testing in PyTest. This is because the application relies heavily on Firebase as the back-end which is something some team members might have limited exposure in. This can cause some friction in the development stage as team members have to learn how to correctly use it to develop the product. \\

We selected Python as our implementation language because all of our team members are familiar with it. However, the original open-source project was written in Java which some of us are not as proficient in. Understanding the Java code and porting it to Python may present some challenges as well, especially if many there exist libraries in Java that do not have a python equivalent.\\

The app would be executable through any desktop system that is capable of running Python scripts. This can include Mac OS, Windows, and Linux to name a few. This ensures portability as the application can be run on several systems without major system level restrictions.\\

We plan to implement some base level functionality for the POC such as login authentication and add/delete/view posts. The authentication modules would take 2 parameters of username and password and send it to the firebase back-end via API requests to authenticate the user credentials. The add/delete/view posts functionality is self explanatory as it will update the database respectively to reflect the commands.

\section{Technology}
\begin{itemize}
  \item Language: We will port the existing project to a Python implementation. The team members are all comfortable with coding in Python. Python also contains several built-in libraries that will be helpful in implementing the project.
  \item IDE: We will use the PyCharm IDE and its included linter to manage the coding style of PEP8.
  \item Testing framework: The testing framework that will be used is PyTest for the unit/integration tests.
  \item Doc Generation: We will be utilizing latex and Doxygen to generate documentation. 
\end{itemize}

\section{Coding style}
We have decided to use the \href{https://www.python.org/dev/peps/pep-0008/}{PEP 8} \textcolor{red}{style guide} to define our code structure and formatting. The linter, outlined above, will help us automatically match this style guide and keep our code clean and readable. Outside of the linting, we will also have to keep our imports organized, use "snake\_case" for functions, "CapCase" for classes and add useful comments to our code.

\section{Project schedule}
The gantt file can also be found in the projectSchedule folder.\\
\textcolor{red}{URL PROVIDED BELOW:}\\
\href{https://gitlab.cas.mcmaster.ca/zamana11/se3xa3-project/-/tree/master/ProjectSchedule)}{https://gitlab.cas.mcmaster.ca/zamana11/se3xa3-project/-/tree/master/ProjectSchedule}
\begin{center}
    \includegraphics[width=1\textwidth]{gantt.png}
\end{center}

\section{Project Review}
\textcolor{red}{The development of this project went as expected. As a team, the work was split up evenly and everyone was able to contribute. The final outcome fulfilled all the functional and non-functional requirements and followed our Gantt chart time frames correctly. The newly redesigned application added many features that expanded far beyond that of the original project functionality.}\\

\textcolor{red}{With that being said, there are a few places that could have been improved. In terms of the final application, security could have been further improved by obscuring the passwords using the * symbol but was not possible due to some technical limitations of the python console outputs. In terms of workflow and time management, everything was completed within planned time slots but could have been finished earlier instead of waiting until the deadline to finalize the work. }

\end{document}